\documentclass[12pt]{report}
\usepackage[a4paper]{geometry} % Configuración de márgenes y dimensiones de página

\usepackage[utf8]{inputenc} % Codificación de caracteres
\usepackage[spanish]{babel} % Configuración de idioma español
\usepackage{amssymb} % Símbolos matemáticos adicionales
\usepackage{subfigure} % Creación de subfiguras dentro de una figura
\usepackage{newunicodechar} % Soporte para caracteres Unicode
\usepackage{graphicx} % Inclusión de imágenes
\usepackage{amsmath} % Ampliación de funcionalidades matemáticas
\usepackage{xspace} % Control del espacio después de comandos
\usepackage{booktabs} % Mejora la apariencia de las tablas
\usepackage[table,xcdraw]{xcolor} % Colores personalizados para tablas
\usepackage{url} % Inclusión y formato de enlaces URL
\usepackage{lineno} % Numeración de líneas
\usepackage{enumitem} % Control de listas y enumeraciones
\usepackage{soul} % Resaltado de texto con subrayado
\usepackage[numbers]{natbib} % Citas bibliográficas y bibliografía
\usepackage[colorlinks=true, bookmarks=false, citecolor=blue, urlcolor=blue, linkcolor=blue, linktoc=page]{hyperref} % Creación de enlaces y referencias
\usepackage{titlesec} % Control del formato de títulos de sección
\usepackage{fancyhdr} % Personalización de encabezados y pies de página
\usepackage{datetime} % Manipulación de fechas y tiempos
% Modificar márgenes
\geometry{left=2.5cm, right=2.5cm, top=2.0cm, bottom=2.0cm}
% Espaciado entre líneas
\linespread{1.5} 
% Cambiar nombre a tablas
\addto\captionsspanish{\renewcommand{\tablename}{Tabla}}
% Cambiar nombre a lista de tablas
\addto\captionsspanish{\renewcommand{\listtablename}{Índice de tablas}}
% Modificar el estilo de los capítulos
\titleformat{\chapter}[display]
{\color{black}\fontfamily{pcr}\LARGE\scshape\raggedleft}
{\chaptertitlename\ \color{red}\fontsize{20}{00}\mdseries\thechapter} {-10pt} {\color{black}\rmfamily\huge}
%\titlespacing*{\chapter}{0pt}{0pt}{50pt}
% Diseño del encabezado y pie de página
\pagestyle{fancy}
\fancyhf{}
\fancyhead[L]{\nouppercase{\leftmark}}
\fancyhead[R]{\nouppercase{\rightmark}}
\fancyfoot[C]{\thepage}
\renewcommand{\headrulewidth}{0.1pt}
% Quitar espacio vertical en listas
\setlist[itemize]{noitemsep, topsep=0pt}
% Numerar las secciones
\setcounter{secnumdepth}{4} % Establece la profundidad de numeración de secciones
\titleformat{\paragraph}{\normalfont\normalsize\bfseries}{\theparagraph}{1em}{} % Define el formato de las subsubsubsecciones
% Modificar la tabla de contenido
\setcounter{tocdepth}{4}
\addto\captionsspanish{\renewcommand{\contentsname}{Contenido}}


%%%%%%%%%%%%%%%%%%%%%%%%%%% AQUI INICIA EL DOCUMENTO %%%%%%%
\begin{document}
\pagenumbering{Roman}
%%%%%%%%%%%%%%%%%%%%%%%%%%% PORTADA %%%%%%%%%%%%%%%%%%%%%%%%
\begin{titlepage}
\newgeometry{left=2.5cm,right=2.5cm,top=2.5cm,bottom=2.5cm}
\begin{minipage}[c]{0.15\textwidth}
  \centering
  \includegraphics[width=\textwidth]{01_Figuras/00_TECNM.png}
\end{minipage}
\hfill % Espacio horizontal entre las cajas
\begin{minipage}[c]{0.8\textwidth}
    \centering
        \textsc{\large Instituto Tecnológico de Pabellón de Arteaga}\\[0.3cm]
        \textsc{\large Ingeniería en TIC's}\\[0.3cm]
        \hrule height1pt
        \vspace{0.1cm}
        \hrule height2pt
        \vspace{0.3cm}
\end{minipage}

\begin{minipage}[c]{0.15\textwidth}
    \vspace{0.5cm}
    \hspace{0.5cm}
    \vrule width1pt height15cm 
    \hspace{0.05cm}
    \vrule width2pt height15cm
\end{minipage}
\hfill % Espacio horizontal entre las cajas
\begin{minipage}[c]{0.8\textwidth}
    \centering
        \textbf{\large Implementacion de Odoo sistema de invnetario y punto de venta}\\[1cm]
        \text{\large Por}\\[1cm]
        \text{\Large Jose Francisco Serna Santana}\\[1cm]
        \textit{\large Avance del proyectó de la implementación de odoo punto de venta e inventario}\\[1cm]
        \textbf{\Large Odoo}\\[1cm]
\end{minipage}
\begin{minipage}[c]{0.20\textwidth}
  \centering
  \includegraphics[width=\textwidth]{01_Figuras/pabellon.png}
\end{minipage}
\hfill % Espacio horizontal entre las cajas
\begin{minipage}[c]{0.8\textwidth}
    \centering
        \mdseries{\large Pabellón de Arteaga, Aguascalientes, \the\year}
\end{minipage}

\end{titlepage}
\restoregeometry
%%%%%%%%%%%%%%%%%%%%%%%%%%% RESUMEN %%%%%%%%%%%%%%%%%%%%%%%%%%%%%
\thispagestyle{empty}
\newgeometry{left=2.5cm,right=2.5cm,top=2.5cm,bottom=2.5cm}
\addcontentsline{toc}{chapter}{Resumen}
\begin{center}
\textsc{\Huge Resumen}\\[0.5cm]
\textbf{\large Implementar un punto de venta basado en la plataforma Odoo para optimizar los procesos de inventario, ventas de productos en ferrteria y abarrotes. Esto permitirá mejorar la eficiencia operativa, reducir costos y tomar decisiones más informadas. }\\[0.25cm]
\text{\large Por: Jose Francisco Serna Santana}\\[0.5cm]
\end{center}


\restoregeometry
%%%%%%%%%%%%%%%%%%%%%%%%%%% TABLA DE CONTENIDO %%%%%%%%%%%%%%%%%
\tableofcontents

\newpage
\pagenumbering{arabic}
\chapter{Introducción}

\large {En este documento se dira el proceso de implementación de Odoo Punto de Venta. El objetivo principal es optimizar nuestras operaciones de venta al por menor, mejorar la eficiencia y precisión en la gestión de inventario y pagos, y brindar una experiencia de compra más satisfactoria a nuestros clientes. }\\[0.25cm]

\newpage
\pagenumbering{arabic}
\chapter{Preguntas}

\large {
1. ¿Cuál es el tamaño aproximado de su inventario actual? 

El tamaño de inventario en abarrotes tenemos un aproximado de 300 productos diferentes y de ferretería son aproximadamente 100 productos.}\\[0.25cm]

\large {
2. ¿Qué tipo de productos vende principalmente?

Los productos que más vendemos de abarrotes son los productos básicos como son leche frijol aceite harina y también lo que son refrescos pan y lo que es dulce, si lo que es fritura.}\\[0.25cm]

\large {
3. ¿Cómo realiza actualmente el seguimiento de su inventario? (¿Manual, hojas de cálculo, otro sistema?)

Es de manera manual de hecho no hemos tenido un inventario así de bien, no lo hemos realizado porque así lo hemos llevado simplemente verdad de una forma, pues básica.}\\[0.25cm]

\large {
4. ¿Con qué frecuencia realiza el inventario físico?

Con respuesta a con qué frecuencia hacemos el inventario físico, pues normalmente lo hacemos semanalmente para ver los productos que nos hacen falta y pues poderlos adquirir o solicitar con el proveedor.}\\[0.25cm]

\large {
5. ¿Cómo captura los datos de sus ventas actualmente?}\\[0.25cm]

\large {
6. ¿Cómo gestiona las devoluciones y los cambios de productos?

Para la gestión de devoluciones lo que hacemos es checar, el producto que venga en buena condición y que no venga manipulado y así pues se hace la devolución.}\\[0.25cm]

\large {
7. ¿Tiene algún sistema de gestión de clientes actualmente?

Actualmente no tengo ninguna gestión de clientes excepto los en mis contactos nada más, pero en sí que tenga una lista para hablar o para decirle a mis clientes tantos productos en sí no tengo.}\\[0.25cm]

\large {
8. ¿Tiene algún sistema de fidelización de clientes o promociones especiales? 

En cuestión del sistema de fidelización de clientes, pues no contamos con ello ya que la mayoría de nuestros clientes, pues son de personal de persona a persona, no, no tenemos productos que vendamos fuera.}\\[0.25cm]

\large {
9. ¿Cuáles son sus principales desafíos actuales en la gestión de su inventario y ventas?

Lo que tenemos actualmente es que en realidad no tenemos un control o un inventario bien para definir y saber cuántos productos en realidad tenemos es lo que el día de hoy no, no contamos con ese con esa situación del inventario.}\\[0.25cm]

\large {
10. ¿Qué características específicas busca en un sistema de punto de venta?

Las características principales que buscamos en un punto de venta es que sea de fácil manejo para que cualquier persona que tenga de empleado pueda manejarlo sin ningún problema.}\\[0.25cm]

\large {
11. ¿Qué tipo de reportes necesita para tomar decisiones sobre su negocio? (¿Ventas por producto, inventario, márgenes de ganancia?)

Bueno, los tipos de reporte que yo ocupo es saber qué productos se mueven más también la venta total, ya sea semanal o mensual de todo el negocio y también el margen, que tanto porcentaje es la ganancia, para poder tomar decisiones.}\\[0.25cm]

\large {
12. ¿Cuántos usuarios necesitarán acceso al sistema? 

Los usuarios que ocuparía es uno administrativo que sería el personal y 6 o 7 que sean para las personas que trabajan en el negocio.}\\[0.25cm]

\large {
13.¿Tiene alguna preferencia en cuanto al diseño de las interfaces del sistema? (Personalización de la apariencia.)

De preferencia que sea un sistema sencillo, fácil de usar con colores y si es posible que maneje iconos para no batallar en la búsqueda de cualquier producto.}\\[0.25cm]

\large {
14. ¿Qué nivel de soporte técnico espera recibir después de la implementación?

Como principio este si se requeriría una capacitación personal para ver y checar todas las inquietudes que pueda tener posteriormente que me elaboren un tutorial que diga de principio a fin el funcionamiento del punto de venta.}\\[0.25cm]

\newpage
\pagenumbering{arabic}
\chapter{Necesidades del Cliente}

· \textbf{Inventario:} No tiene un sistema formal para llevar el inventario, lo hace de manera manual.

· \textbf{Ventas:} No captura los datos de ventas de manera sistemática.

· \textbf{Gestión de clientes:} No tiene un sistema de gestión de clientes ni de fidelización.

· \textbf{Reportes:} Necesita reportes simples de ventas por producto, ventas totales y márgenes de ganancia.

· \textbf{Facilidad de uso:} Busca un sistema intuitivo y fácil de usar, idealmente con iconos.

· \textbf{Capacitación:} Requiere capacitación y un tutorial detallado.


\newpage
\pagenumbering{arabic}
\chapter{Propuesta de Valor de Odoo}

Odoo puede ofrecer una solución  a las necesidades del cliente, proporcionando un sistema de punto de venta (POS) que:

· \textbf{Simplifica la gestión del inventario:} Permite un seguimiento preciso del stock en tiempo real, automatiza los pedidos de reposición y reduce las pérdidas por productos caducados.

· \textbf{Optimiza las ventas:} Facilita la captura de datos de ventas, genera tickets y facturas electrónicas, y permite realizar promociones y descuentos.

· \textbf{Mejora la relación con los clientes:} Ofrece herramientas para gestionar la base de clientes, crear programas de fidelización y enviar campañas de marketing.

· \textbf{Proporciona información valiosa:} Genera reportes personalizados sobre ventas, inventario, márgenes de ganancia y otros indicadores clave de rendimiento.

· \textbf{Es fácil de usar:} Cuenta con una interfaz intuitiva y personalizable, ideal para usuarios con poca experiencia en sistemas informáticos.


\end{document}
